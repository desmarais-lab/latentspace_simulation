
\section{Response Memo}\label{response-memo}

\subsection{Response to Reviewer 1}\label{response-to-reviewer-1}

\begin{quote}
``The paper does not accurately characterize or analytically understand
the latent space approaches employed by the work that is cited. For
example Dorff and Ward (2013) do not use LATENTNET. LATENTNET is
extremely outdated and is not the software used in much of the other
work cited herein. This undermines much of the criticisms in the papers
and raises a red flag that suggests the authors do not grasp the
modeling approaches as deeply as they would present.''
\end{quote}

While we appreciate that the \texttt{latentnet} package is not the
implementation most commonly used in political science, it is used
outside of political science. We have made this point more clearly in
the revised manuscript. Most importantly, our criticism is not specific
to the implementation in \texttt{latentnet}, and we argue (again, more
clearly in the revised manuscript), that our argument does not depend on
the specific software implementation or parameterization used, and hence
this criticism does not speak to our point, which is that that the LSM
cannot adjust for unobserved confounders, even in a synthetic scenario
designed to make this as easy as possible.

We updated the Problem section of the manuscript to represent, (1) that we know about extensions to the LSM, (2) to be clear that we think the problem affects all versions of the LSM, and (3) to quote from AMEN to echo the problem we see with using LSM as a method of adjusting for confounders. We cannot possibly analyze all of the individual variants of the LSM, but our point is more broadly directed at the use of latent variables to adjust for confounding network dependencies. 

\begin{quote}
``The authors are not up-to-date on the state of latent modeling in the
field and have ignored more recent work (which addresses many of the
concerns raised by the authors) particularly by Peter Hoff and Shahryar
Minhas. Additionally, the progress made by the newly developed AMEN
software is completely overlooked by this article.''
\end{quote}

We have added discussion of the AMEN package, which we were aware of,
but again, does not apply to our argument for the same reason.

\begin{quote}
``In that vein, authors identify a problem (e.g., collinearity between
latent distance parameters and exogenous variables in the latentnet
framework) and do not solve it nor discuss any recent work that tries to
address this problem.''
\end{quote}

We added discussion of the way in which AMEN addresses the problem of correlation between the observed covariates and the latent factors. We point out, however, that the AMEN approach is designed to address Type 2 error by prioritizing the observed covariates and using the latent factors to explain the variation in the residuals. This does not address Type 1 error (which is more troubling than Type 2), and probably makes Type 1 worse.


Our point is that when the measured covariates are \emph{not}
independent of the unobserved structure in the network, that estimating
that structure using latent variables cannot adjust for any confounding.
We are unaware of any work that attempts to tackle this problem and we
furthermore consider the problem as we've posed it intractable for
theoretical reasons, which we have elucidated more clearly in our
revised version. In generally latent variables cannot be used to adjust
the effects of measured covariates. In the case where the measured
covariates are independent of any unmeasured structure, the LSM does no
worse than standard models, as we show in our simulation.

\subsection{Response to Reviewer 2}\label{response-to-reviewer-2}

\begin{quote}
``Yet, in their efforts to contextualize their research the authors seem
to disregard or not be aware of notable differences between existing
latent variable approaches. Specifically, much of the work they cite
(e.g., Ward Siverson and Cao 2007, Ward, Ahlquist and Rozenas 2013,
etc.) does not even use the type of latent variable model that the
authors focus on in their paper. In fact, with the exception of a 2012
piece by Kirkland, I am not aware of any published political science
work that actually uses the latent space model for network analysis the
authors study here.''
\end{quote}

While this is true, it is orthogonal to our point, which is that the
LSM, regardless of how the latent variables are parameterized, cannot
adjust for unobserved network confounders. We chose the first
formulation of the LSM to make our synthetic demonstration as
understandable as possible. We argue that it shows that even in the most
ideal scenario, the LSM cannot adjust for confounders that are
unmeasured. We have made the point that this behavior is independent of
the parameterization of the latent variables more clear in the revised
manuscript. While we certainly could have chosen any number of the
variations of the LSM for our simulation, absent clear reasons to do so
we do not believe that it is reasonable to expend the substantial
computational resources necessary to expand the simulation.

\begin{quote}
``The difference between these approaches has been shown to be
consequential in a number of works. Hoff \& Ward (2004) and Hoff (2005)
detail the ways in which the GBME is different than the latent distance
model that is employed in the latentnet package. Hoff (2008) explicitly
discusses how the latent distance model confounds transitivity and
stochastic equivalence (two types of third order interdependencies) in
problematic ways that the latent factor model is able to avoid. The
issue this paper seems most concerned with showing is that the latent
distance model does not solve the problem of bias and inferential error
relative to GLM for observed covariates. They do so through a well
executed simulation exercise and the results are certainly compelling in
favor of showcasing a shortcoming of the latent distance model. At the
same time, this result is not very surprising. For the latent distance
model we expect there to be some collinearity between covariates for
which there is homophily and latent position. In theory, a weak prior on
the covariate coefficients and a strong (tight) prior on the
coefficients should minimize this effect, but at the cost of less
expressiveness for latent positions.''
\end{quote}

We agree that the parameterization of the latent variables can affect
what can be learned about omitted structure in the network, as
suggested. However, this is unrelated to our point which is that it is
not possible to adjust the estimates of the effect of observed
covariates using a latent variable model that can represent unobserved
structure. We argue more clearly in the revised manuscript that this is
not a property of the Euclidean distance formulation we chose for
expository simplicity.
