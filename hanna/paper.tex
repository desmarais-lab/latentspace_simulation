\documentclass[11pt]{article}
\usepackage[pdftex]{graphics}
\usepackage{amsmath,amssymb,rotating,multirow}
\usepackage{times}
\usepackage{url}

\usepackage{booktabs}

\usepackage{subfigure}

\usepackage[longnamesfirst]{natbib}

\usepackage[plain]{algorithm}
\usepackage{algpseudocode}

%\usepackage{parskip}
\usepackage{setspace}

\usepackage{geometry}
\geometry{verbose,letterpaper}

\setlength{\parskip}{0pt}

\renewenvironment{quote}{\list{}{\listparindent=1.5em\itemindent=\listparindent\rightmargin=1.65cm\leftmargin=1.65cm}\item\relax}{\endlist}

\newcommand{\comment}[1]{}

\title{Inferring the Effects of Observed Covariates in\\ Latent Space
  Models for Networks}

\author{Zachary Jones, Matthew Denny, Bruce Desmarais, Hanna Wallach}
\date{\today}

\bibliographystyle{chicago}

\begin{document}

\singlespacing
\maketitle

\thispagestyle{empty}

% TODO: Consider rewriting the abstract to talk about type I vs. type
% II error and the fact that only type II error has been addressed.

\begin{abstract}
  \noindent Due to the complex dependencies found in network data,
  scholars draw upon an increasingly sophisticated toolkit for
  statistical inference. The latent space model (LSM) models the edges
  in a network via a combination of the generalized linear model (GLM)
  and a latent spatial embedding of the network's nodes. In many
  applications, researchers have assumed that this embedding can
  control for unmeasured confounding network structure. However, there
  has been little research that considers whether this is indeed the
  case. Via a simulation study, we investigate the LSM's ability to
  control for unmeasured confounding network structure. We find that
  under even moderate confounding, the LSM does not exhibit lower
  estimation error or inference error than the GLM; it does, however,
  exhibit substantially lower prediction error. We therefore conclude
  that the LSM is most appropriate for predictive or exploratory
  analyses.
\end{abstract}

% latent spatial embedding of the network's nodes

% complex dependencies
% dependencies such as reciprocity, transitivity, and homophily

% unmeasured counfounding variables
% unmeasured confounding network structure

% estimation error
% biased estimates of covariate effects

% inference error
% errors in hypothesis testing

% prediction error

% exploratory analyses, inferential analysis, predictive analysis

% observed covariates
% observed covariates of interest

% NOTE: ``confounding variables'' are correlated with both the
% dependent (edge values) and independent variables (covariates).

% NOTE: ``unmeasured confounding variables'' means that the
% confounding variables are omitted from the model.

% NOTE: Controlling for unmeasured confounding variables often means
% measuring them and including them in the model.

\section{Introduction}

% LSM wasn't developed to account for unmeasured confounding network
% structure, so should NOT talk about it in that same sentence. TODO:
% Rewrite to say that although it wasn't developed for this purpose,
% the LSM is often assumed to control for confounding variables.

% Regarding this, Bruce says: ERGM and SAOM were designed for fully
% specifying the network model, including how covariates effect the
% edges, and how edges relate to each other. ERGM and SAOM do not
% involve any latent variables. In other words, when the analyst is
% using ERGM or SAOM (s)he is assuming to know the correct form of the
% model, and measure all of the data needed to estimate the model.

% The ``unmeasured'' part does not apply to ERGM and SAOM. All of
% these models were designed to account for structural predictors of
% tie formation (e.g., you and I are tied because we have a mutual
% friend---also in the network---who introduced us, not because of our
% background covariates). In ERGM and SAOM you need to explicitly
% specify the functional form of these dependencies in order to
% account for them. By specifying them exactly they are measured on
% the network while fitting the model. The downside is that ERGM does
% not account for any dependencies that are not specified by the
% researcher. With the LSM, the big matrix of latent variables is
% supposed to absorb these dependencies, as well as omitted features.

Inferential analyses of network data have grown increasingly
sophisticated in recent years. Scholars are well-versed in the risks
associated with ignoring unmeasured confounding network structure. If
not accounted for, dependencies such as reciprocity, transitivity, and
homophily can lead to biased estimates of covariate effects and to
hypothesis testing errors, in much the same way that omitted-variable
bias can affect conventional regression analyses
\citep{ward07disputes,kinne14dependent,cranmer17critique,hays10spatial}. Researchers
have therefore proposed a number of statistical models to account for
unmeasured confounding network structure, including the exponential
random graph model \citep[ERGM;
  e.g.,][]{lazer10coevolution,cranmer11inferential,desmarais12micro-level},
the stochastic actor-oriented model \citep[SAOM;
  e.g.,][]{berardo10self-organizing,kinne14dependent}, and the latent
space model \citep[LSM;
  e.g.,][]{ward07disputes,ward07persistent,kirkland12multimember}.

The ERGM and the SAOM both account for dependencies using an approach
similar to that used to control for unmeasured confounding variables
in regression analyses. The researcher specifies a set of dependencies
that (s)he hypothesizes to be important to the network. These
dependencies are then explicitly included in a model that
simultaneously represents the observed covariates of interest. In
contrast, the LSM takes a different approach and uses a latent spatial
embedding to control for unmeasured confounding network structure. The
LSM therefore has an advantage over the ERGM and the SAOM in that the
researcher does not need to hypothesize a set of dependencies that may
be important. However, this advantage is contingent upon the LSM's
ability to identify confounding network structure that would otherwise
be attributed to the observed covariates.

Despite the growing popularity of these models, there have been few
studies that investigate their performance at accounting for
unmeasured confounding network structure. In this article, we focus on
the LSM and study its ability to reduce estimation error and inference
error using a latent spatial embedding of network structure.


TODO: Something about type I error.

\section{The Latent Space Model}

What the LSM is and why it was developed.

What the LSM is used for (currently in 1, 1.1, and 2).

Concerns re. its use, type I and II error, type II addressed.

\section{}


\section*{Acknowledgements}

This work was supported in part by National Science Foundation (NSF)
grants DGE-1144860, SES-1558661, SES-1619644, and CISE-1320219. Any
opinions, findings, and conclusions or recommendations are those of
the authors and do not necessarily reflect those of the sponsor.

\bibliography{references}

\end{document}
